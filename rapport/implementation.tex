section{Implementation}
subsection{Choix de Oracle}
subsection{implémentation de la base}
Après la conception et la modélisation de la structure de la base, nous nous sommes intéressés à la création des tables de la base et des données. 
Lors de la création des tables, il fallu suivre un certain ordre respectant les contraintes d'intégrité. Ainsi, chaque clé étrangère doit référencier 
une table qui existe déjà. En fait, pour la mise en place de la base, nous avons créé 3 fichiers base.sql, autoicrement.sql et donnees.sql. Le premier permet de créer les tables et de les effacer. Le deuxième permet de créer des séquences. Le dernier permet 
le remplissage des tables avec des lignes significatives qui vont nous servir pour tester les requêtes. 

Création
Dans le fichier autoincrement.sql, nous avons créé les séquences et les déclencheurs. Elles correspondent aux clés qui s'auto-incrémentent, puisque Oracle ne reconnaît pas le mot clé auto_increment, contrairement à MySQL.
---exemple---

Dans le fichier base.sql, nous avons créé les tables tout en respectant les contraintes référentielles iter-tables. Voici un exemple de création de table : 

create table CLUB
(
    NUMERO_CLUB                   NUMBER(3)			 not null,
    NOM_CLUB                      CHAR(20)               not null,	 
    constraint pk_club primary key (NUMERO_CLUB)
);

Voici un exemple de suppression de table :
----exemple------

Dans le fichier donnees.sql, nous avons rempli les tables avec des données significatives pour le test des différentes requêtes. Cela se fait à l'aide du mot clé insert. 
Voici un exemple d'insertion de données dans une table :
----Exemple-----

subsection{Les commandes SQL}
Cette partie du travail nous a permis de tout tester au début: consultation, mise à jour, recherche, suppression, ajout, requêtes de statistiques etc. 
chaque commande SQL correspond bien à un besoin réel, soit pour la gestion interne du club, soit pour le besoin de l'utilisateur.

subsubsection{Ajout, suppression, mise à jour}
Toutes ces requêtes se trouvent dans le fichier update.sql. Elles permettent d'ajouter, supprimer ou mettre à jour des données en fonction du désir de l'utilisateur.
Voici un exemple d'ajout de données d'un joueur:

insert into JOUEUR values(21,'MONTAND','YVES','13-OCT-89','TALENCE', '14-JAN-01',1);
commit;

Voici un exemple de suppression des données d'un club:
delete from club where club.numero_club = 10;
commit;

Voici un exemple de mise à jour du score d'une équipe:
update rencontre set rencontre.score_equipe1_rencontre = 15 where rencontre.numero_equipe1 = 3;
commit;

subsubsection{Consultation}
Ces requêtes sont réparties dans les fichiers recherche.sql et consultation.sql. Le premier fichier permet à l'utilisateur de chercher une entité en fonction d'une donnée rentrée par l'utilisateur.
Voici l'exemple de recherche d'un entraîneur par son nom :
select * from ENTRAINEUR where NOM_ENTRAINEUR = 'DUBOIS';

Voici l'exemple de recherche des dates de rencontres auxquelles a participé un joueur donné (l'utilisateur doit fournir le nom du joueur) :
select DATE_RENCONTRE from RENCONTRE where NUMERO_RENCONTRE in (select NUMERO_RENCONTRE from PARTICIPE where NUMERO_LICENCE = (select NUMERO_LICENCE from JOUEUR where NOM_JOUEUR = 'GARCIA'));

Le fichier consultation.sql permet d'afficher toutes les informations sur les clubs, les équipes et les joueurs. 
L'exemple suivant permet de consulter toutes les information de toutes les équipes :
select equipe.numero_equipe, equipe.nom_equipe, club.nom_club, categorie.nom_categorie
from equipe, club, categorie
where equipe.numero_club = club.numero_club and equipe.numero_categorie = categorie.numero_categorie;

L'exemple qui suit permet de consulter les scores des matchs joués à une date donnée :
select rencontre.score_equipe1_rencontre as Equipe1 , rencontre.score_equipe2_rencontre as Equipe2
from rencontre
where rencontre.date_rencontre = '21-FEB-87'; 

L'utilisateur peut également consulter le nombre de matchs gagnés, perdus et nuls pour un club donné.
Pour le club numéro 2 voici les trois requêtes que nous avons utilisées :
nombre de matchs gagnés :
select count(*) as GAGNER
from(
(select *
from club, equipe, rencontre
where equipe.numero_equipe = rencontre.numero_equipe1 
      and rencontre.score_equipe1_rencontre > rencontre.score_equipe2_rencontre
      and equipe.numero_club = club.numero_club
      and club.numero_club = 2)
union
(select *
from club, equipe, rencontre 
where equipe.numero_equipe = rencontre.numero_equipe2
      and rencontre.score_equipe2_rencontre > rencontre.score_equipe1_rencontre
      and equipe.numero_club = club.numero_club
      and club.numero_club = 2)
);

nombre de matchs perdus:
select count(*) as PERDU
from(
(select *
from club, equipe, rencontre
where equipe.numero_equipe = rencontre.numero_equipe1 
      and rencontre.score_equipe1_rencontre < rencontre.score_equipe2_rencontre
      and equipe.numero_club = club.numero_club
      and club.numero_club = 2)
union
(select *
from club, equipe, rencontre 
where equipe.numero_equipe = rencontre.numero_equipe2
      and rencontre.score_equipe2_rencontre < rencontre.score_equipe1_rencontre
      and equipe.numero_club = club.numero_club
      and club.numero_club = 2)
);

nombre de matchs nuls :
select count(*) as NULS
from(
(select *
from club, equipe, rencontre
where equipe.numero_equipe = rencontre.numero_equipe1 
      and rencontre.score_equipe1_rencontre = rencontre.score_equipe2_rencontre
      and equipe.numero_club = club.numero_club
      and club.numero_club = 2)
union
(select *
from club, equipe, rencontre 
where equipe.numero_equipe = rencontre.numero_equipe2
      and rencontre.score_equipe2_rencontre = rencontre.score_equipe1_rencontre
      and equipe.numero_club = club.numero_club
      and club.numero_club = 2)
);


subsubsection{Les statistiques}
Ces requêtes se trouvent dans le fichier statistques.sql. Elle permettent de réaliser les requêtes de statistiques demandées dans l'énoncé du sujet.
Pour la moyenne des points marqués à une date donnée (ici c'est 27/02/2006:

select avg(participe.cumul_points_marques_joueur) as MOYENNE_POINTS
from participe, rencontre
where rencontre.date_rencontre = '21-FEB-06'
      and rencontre.numero_rencontre = participe.numero_rencontre;

La commande suivante permet de consulter la moyenne des points marqués depuis le début de la saison 2001 :

select avg(participe.cumul_points_marques_joueur) as MOYENNE_POINTS
from joueur, participe, rencontre
where rencontre.date_rencontre > '01-JAN-01'
      and rencontre.numero_rencontre = participe.numero_rencontre;

L'exemple suivant permet d'avoir le classement des meilleurs joueurs de la catégorie cadet pour la date du 21/02/07 :
select joueur.numero_licence, sum(participe.cumul_points_marques_joueur) as SCORE
from joueur, participe, rencontre, equipe
where equipe.numero_categorie = 1
      and rencontre.date_rencontre = '21-FEB-07'
      and equipe.numero_equipe = joueur.numero_equipe
      and joueur.numero_licence = participe.numero_licence
      and participe.numero_rencontre = rencontre.numero_rencontre
group by joueur.numero_licence 
order by score DESC;

L'exemple qui suit nous permet d'avoir le classement des équipes en fonctions de leurs scores durant les rencontres auxquelles elles avaient participé:
select num_equipe, nom_equipe, sum(score) as Score
from(
(select rencontre.numero_equipe1 as NUM_EQUIPE , rencontre.nom_equipe1 as NOM_EQUIPE ,rencontre.score_equipe1_rencontre as SCORE
from rencontre)
union
(select rencontre.numero_equipe2 as NUM_EQUIPE , rencontre.nom_equipe2 as NOM_EQUIPE ,rencontre.score_equipe2_rencontre as SCORE
from rencontre))
group by NUM_EQUIPE, NOM_EQUIPE
order by Score DESC;


section*{Conclusion}
Au cours de ce projet, nous avons pu mettre en oeuvre les connaissances acquises au cours des séances de cours et de TD en vue de la création d'une base de données se voulant aussi proche de la réalité que possible.
Ce sujet, qui à première vue semblait simple, s'est avéré plus difficile qu'il n'y paraissait -. En effet, les liens entre associations et entités semblaient être évidents, sont apparus plus compliqués, car il fallait tenir compte des redondances dans la base de données, qu'il faut éviter le plus souvent possible.
La majeure partie du temps que nous avons consacré dans le projet a été dans le but d'établir la modélisation des données.
Outre la mise en pratique des connaissances en SGBD, ce projet nous a permis de découvrir comment réaliser une interface graphique en java.
en somme, ce projet nous a permis de mieux comprendre la complexité qu'engendre la gestion des bases de données : ses buts, ses problèmes et ses difficultés. 