Pour établir l'interface graphique de base des données, nous avons utilisé JDBC avec les bibliothèques JFrame, JPanel, JButton, JTable etc. L'interface s'organise en cinq zones principales: le titre, les options, le tableau, les statistiques, et la mise à jour. Pour chaque partie, nous avons utilisé un objet JPanel pour organiser les postions des différentes parties dans l'interface.

Dans la zone des options, il est possible de sélectionner un bouton. Chaque bouton retourne un tableau de données different, par exemple, quand nous cliquons sur le bouton "Joueur", et il envoie une commande "select * from JOUEUR" et affiche les toutes informations sur les joueurs.

%% ajouter 1.png

Pour la zone des statistiques, nous avons proposé quarte méthodes, et avons consacré un bouton pour chaque méthode. Quand nous appuions sur un bouton de statistique, il affiche une nouvelle fenêtre, dans laquelle nous pourrons remplir les conditions de recherche. Pour valider la requête, il suffit de cliquer sur le bouton "Find". Cela permet d'envoyer une commande et de contruire un nouveau tableau. Quand on selectionne le Bouton "Clear", les chiffres que nous avons rentrés seront supprimés. Quand on selectionne le Bouton "Cancel", on rentoure à la fenêtre principale, et aucune commande n'est exécutée.

%% ajouter 2.png

Par exemple, si on veut consulter le classement des meilleurs joueurs d'une journée pour une catégorie, nous devons remplir le numéro de catégorie et la date de rencontre, et après l'interface doit envoyer une commande :

%% ajouter 3.png

La dernière zone est dédiée pour mettre à jour des données. Nous pourrons ajouter, supprimer ou modifier des données. Quand on appuie sur l'un des boutons, une nouvelle fenêtre s'ouvre avec un tableau au milieu. Dans ce tableau, nous pourrons modifier des données et à la fin on appuie sur add element ou update ou delete element.
%% ajouter 4.png


Après avoir conçu l'interface graphique, nous n'avons pas réussi à afficher la fenêtre tout en se connectant à Oracle. Nous avons quand même gardé l'interface graphique mais nous nous sommes intéressés par la suite à un affichage dans le terminal, puisqu'il n'est pas obligatoire de réaliser une interface et que le but de ce projet et de mettre en pratique les connaissances acquises au cours des séances de cours et de TD de SGBD. 
